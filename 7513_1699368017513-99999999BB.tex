
\documentclass[a4paper,12pt]{article}

\usepackage{amsmath}

\usepackage{amssymb}

\usepackage{graphicx}

\usepackage[margin=1in]{geometry}

\setlength{\parindent}{0pt}

\graphicspath{/home/shizlong/spok/pdftolatex/localstore/7513_1699368017513-99999999BBassets}

\begin{document}

\begin{figure}[h]

\includegraphics[width=\textwidth]{/home/shizlong/spok/pdftolatex/localstore/7513_1699368017513-99999999BBassets/0.jpg}

\centering

\end{figure}

    AUSPOS GPS Processing Report 

\vspace{10pt}

    November 7, 2023 

\vspace{10pt}

    This document is a report of the GPS data processing undertaken by the AUSPOS Online  GPS Processing Service (version: AUSPOS 2.4) . The AUSPOS Online GPS Process-  ing Service uses International GNSS Service (IGS) products (final, rapid, ultra-rapid  depending on availability) to compute precise coordinates in International Terrestrial  Reference Frame (ITRF) anywhere on Earth and Geocentric Datum of Australia (GDA)  within Australia. The Service is designed to process only dual frequency GPS phase data. 

\vspace{10pt}

    An overview of the GPS processing strategy is included in this report. 

\vspace{10pt}

    Please direct any correspondence to GNSSAnalysisQga.gov.au 

\vspace{10pt}

    Geoscience Australia   Cnr Jerrabomberra and Hindmarsh Drive  GPO Box 378, Canberra, ACT 2601, Australia  Freecall (Within Australia): 1800 800 173   Tel: +61 2 6249 9111. Fax +61 2 6249 9929  Geoscience Australia   Home Page: http://www.ga.gov.au 

\vspace{10pt}

         AUSPOS 2.4 Job Number: \# 7513 1 ©Commonwealth of Australia    ser: panzhin@igdure . .  User: panzhin@igduran.ru (Geoscience Australia) 2023 

\vspace{10pt}

\par

\vspace{10pt}

\begin{figure}[h]

\includegraphics[width=\textwidth]{/home/shizlong/spok/pdftolatex/localstore/7513_1699368017513-99999999BBassets/1.jpg}

\centering

\end{figure}

\begin{figure}[h]

\includegraphics[width=\textwidth]{/home/shizlong/spok/pdftolatex/localstore/7513_1699368017513-99999999BBassets/2.jpg}

\centering

\end{figure}

    All antenna heights refer to the vertical distance from the Ground Mark to the Antenna  Reference Point (ARP). 

\vspace{10pt}

                   Station (s) Submitted File Antenna Type Antenna Start Time End Time   Height (m)  NVKZ NVKZ204.rnx NONE NONE 0.000 2021/07/23 00:00:30 2021/07/23 23:59:30  POLY POLY204.rnx NONE NONE 0.000 2021/07/23 00:00:30 2021/07/23 23:59:30                                   

\vspace{10pt}

    2 Processing Summary 

\vspace{10pt}

\begin{figure}[h]

\includegraphics[width=\textwidth]{/home/shizlong/spok/pdftolatex/localstore/7513_1699368017513-99999999BBassets/3.jpg}

\centering

\end{figure}

\begin{figure}[h]

\includegraphics[width=\textwidth]{/home/shizlong/spok/pdftolatex/localstore/7513_1699368017513-99999999BBassets/4.jpg}

\centering

\end{figure}

         AUSPOS 2.4 Job Number: \# 7513 2 ©Commonwealth of Australia    ser: panzhin@igdure . .  User: panzhin@igduran.ru (Geoscience Australia) 2023 

\vspace{10pt}

\par

\vspace{10pt}

\begin{figure}[h]

\includegraphics[width=\textwidth]{/home/shizlong/spok/pdftolatex/localstore/7513_1699368017513-99999999BBassets/5.jpg}

\centering

\end{figure}

\begin{figure}[h]

\includegraphics[width=\textwidth]{/home/shizlong/spok/pdftolatex/localstore/7513_1699368017513-99999999BBassets/6.jpg}

\centering

\end{figure}

    All coordinates are based on the IGS realisation of the ITRF2014 reference frame. All  the given ITRF2014 coordinates refer to a mean epoch of the site observation data. All  coordinates refer to the Ground Mark. 

\vspace{10pt}

    3.1 Cartesian, ITRF2014                        Station X () Y (m) Z (mn) ITRF2014 @  NVKZ 189708 .580 3775106.051 5120442.971 23/07/2021  POLY 239370 .329 3695874 .222 5175528 .204 23/07/2021  ARTU 1843956. 322 3016203. 264 5291261 .803 23/07/2021  BJFS -2148744.611 4426641.151 4044655 .786 23/07/2021  CHUM 1228950.345 4508080.012 4327868 .541 (23/07/2021  JFNG -2279829.137 5004706 .440 3219777 .369 23/07/2021  KIT3 1944944 .688 4556652. 368 4004326 .064 23/07/2021  LHAZ -106942.210 5549269.752 3139215 .235 (23/07/2021  MDVJ 2845455 .726 2160954. 447 5265993 . 330 23/07/2021  NOVM 452260.671 3635877 .608 5203453 .339 23/07/2021  NRIL 64536 .821 2253782.911 5946363.514 23/07/2021  TASH 1695944.763 4487138 .677 4190140.757 23/07/2021  URUM 193030.121 4606851 . 287 4393311.526 23/07/2021  YAKT -1914999.311 2308241 .445 5610225 .487 23/07/2021     

\vspace{10pt}

    3.2 Geodetic, GRS80 Ellipsoid, ITRF2014 

\vspace{10pt}

    Geoid-ellipsoidal separations, in this section, are computed using a spherical harmonic  synthesis of the global EGM2008 geoid. More information on the EGM2008 geoid can be  found at http: //earth-info.nga.mil/GandG/wgs84/gravitymod/egm2008/. 

\vspace{10pt}

         Station Latitude Longitude Ellipsoidal Derived Above                   (DMS) (DMS) Height (m) Geoid Height (m)  NVKZ 53 44 57.31589 87 O7 23.38641 189.459 229.756  POLY 54 35 39.64569 86 17 39.49974 216.645 255.466  ARTU 56 25 47.36092 58 33 37.66871 247 .582 253.906  BJFS 39 36 30.95878 115 53 32.96910 87.458 97.499  CHUM 42 59 54.60551 74 45 03.97494 716.343 759 .333  JFNG 30 30 56.03223 114 29 27.68024 71.298 84.708  KIT3 39 08 05.16356 66 53 07.62242 622.486 659.583  LHAZ 29 39 26.40351 91 06 14.52009 3624.613 3659. 304  MDVJ 56 01 17.37864 37 12 52.23611 257.113 241.418  NOVM 55 01 49.80337 82 54 34.17953 150.085 186.316  NRIL 69 21 42.59901 88 21 35.24393 47.940 62.034  TASH 41 19 40.97914 69 17 44.05751 439.702 483.272  URUM 43 48 28.61973 87 36 02.42049 858.865 922.244    YAKT 62 01 51.44953 129 40 49.11316 103.404 108.870     

\vspace{10pt}

         AUSPOS 2.4 Job Number: \# 7513 3 ©Commonwealth of Australia    ser: panzhin@igdure . .  User: panzhin@igduran.ru (Geoscience Australia) 2023 

\vspace{10pt}

\par

\vspace{10pt}

\begin{figure}[h]

\includegraphics[width=\textwidth]{/home/shizlong/spok/pdftolatex/localstore/7513_1699368017513-99999999BBassets/7.jpg}

\centering

\end{figure}

    3.3 UTM Grid, GRS80 Ellipsoid, ITRF2014 

\vspace{10pt}

         Station East North Zone Ellipsoidal Derived Above                   (m) (m) Height (m) Geoid Height (m)  NVKZ 508121.708 5955631.214 45 189.459 229.756  POLY 454403 .848 6049881 .044 45 216.645 255.466  ARTU 596235. 186 6255011.365 40 247 .582 253.906  BJFS 404924 .742 4384902 .947 50 87.458 97.499  CHUM 479712.412 4760678 .445 43 716.343 759 .333  JFNG 259237 .975 3378594. 108 50 71.298 84.708  KIT3 317236.791 4333861 .156 42 622.486 659.583  LHAZ 316496 . 230 3282318 .874 46 3624.613 3659 .304  MDVJ 388711.455 6209909 .948 37 257.113 241.418  NOVM 622047 .134 6099852. 355 44 150.085 186.316  NRIL 553484 .953 7695302 .993 45 47.940 62.034  TASH 524734 .375 4575216 .868 42 439.702 483.272  URUM 548313.480 4850717 .938 45 858.865 922.244    YAKT 535596. 120 6877815 .579 52 103.404 108.870     

\vspace{10pt}

    3.4 Positional Uncertainty (95\% C.L.) - Geodetic, ITRF2014 

\vspace{10pt}

         Station Longitude(East) (m) Latitude(North) (m)  Ellipsoidal Height(Up) (m)                   NVKZ 0.005 0.004 0.009  POLY 0.005 0.004 0.009  ARTU 0.006 0.004 0.010  BJFS 0.006 0.004 0.010  CHUM 0.005 0.004 0.008  JFNG 0.006 0.005 0.014  KIT3 0.006 0.004 0.011  LHAZ 0.006 0.005 0.012  MDVJ 0.006 0.005 0.013  NOVM 0.005 0.004 0.008  NRIL 0.005 0.005 0.011  TASH 0.005 0.004 0.009  URUM 0.005 0.004 0.010  YAKT 0.006 0.005 0.011     

\vspace{10pt}

         AUSPOS 2.4 Job Number: \# 7513 4 ©Commonwealth of Australia    ser: panzhin@igdure . .  User: panzhin@igduran.ru (Geoscience Australia) 2023 

\vspace{10pt}

\par

\vspace{10pt}

\begin{figure}[h]

\includegraphics[width=\textwidth]{/home/shizlong/spok/pdftolatex/localstore/7513_1699368017513-99999999BBassets/8.jpg}

\centering

\end{figure}

    4 Ambiguity Resolution - Per Baseline 

\vspace{10pt}

                        Baseline Ambiguities Resolved Baseline Length (km)  LHAZ - URUM 70.0 \% 1597.152  CHUM - NVKZ 80.5 \% 1498 .482  KIT3 - TASH 81.5 \% 318.371  NOVM - NVKZ 81.5 \% 308.559  CHUM - URUM 76.2 \% 1042.674  POLY - NVKZ 89.2 \% 108.528  NVKZ - YAKT 22.6 \% 2611.776  ARTU - NVKZ 79.5 \% 1828.018  BJFS - JFNG 93.9 \% 1015.759  BJFS - POLY 87.8 \% 2741.530  NRIL - POLY 93.4 \% 1644.500  CHUM - TASH 87.5 \% 487 .331  ARTU - MDVJ 71.0 \% 1317 .228  AVERAGE 78.1\% 1270.762     

\vspace{10pt}

    Please note for a regional solution, such as used by AUSPOS, ambiguity resolution success  rate of 50\% or better for a baseline formed by a user site indicates a reliable solution. 

\vspace{10pt}

         AUSPOS 2.4 Job Number: \# 7513 5 ©Commonwealth of Australia    ser: panzhin@igdure . .  User: panzhin@igduran.ru (Geoscience Australia) 2023 

\vspace{10pt}

\par

\vspace{10pt}

\begin{figure}[h]

\includegraphics[width=\textwidth]{/home/shizlong/spok/pdftolatex/localstore/7513_1699368017513-99999999BBassets/9.jpg}

\centering

\end{figure}

    5 Computation Standards 

\vspace{10pt}

\begin{figure}[h]

\includegraphics[width=\textwidth]{/home/shizlong/spok/pdftolatex/localstore/7513_1699368017513-99999999BBassets/10.jpg}

\centering

\end{figure}

    5.2 Data Preprocessing and Measurement Modelling         Data preprocessing    Phase preprocessing is undertaken in a baseline by baseline  mode using triple-difference. In most cases, cycle slips are  fixed by the simultaneous analysis of different linear combi-  nations of L1 and L2. If a cycle slip cannot be fixed reliably,  bad data points are removed or new ambiguities are set up A  data screening step on the basis of weighted postfit residuals  is also performed, and outliers are removed.         Basic observable    Carrier phase with an elevation angle cutoff of 7° and a sam-  pling rate of 3 minutes. However, data cleaning is performed  a sampling rate of 30 seconds. Elevation dependent weight-  ing is applied according to 1/sin(e)? where e is the satellite  elevation.         Modelled observable    Double differences of the ionosphere-free linear combination.         Ground antenna  phase centre calibra-  tions    1GS14 absolute phase-centre variation model is applied.         Tropospheric Model    A priori model is the GMF mapped with the DRY-GMF.         Tropospheric Estima-  tion    Zenith delay corrections are estimated relying on the WET-  GMF mapping function in intervals of 2 hour. N-S and E-W  horizontal delay parameters are solved for every 24 hours.              Tropospheric | Map- | GMF  ping Function  Ionosphere First-order effect eliminated by forming the ionosphere-free    linear combination of L1 and L2. Second and third effect  applied.         Tidal displacements    Solid earth tidal displacements are derived from the complete  model from the IERS Conventions 2010, but ocean tide load-  ing is not applied.         Atmospheric loading    Applied         Satellite centre of  mass correction    IGS14 phase-centre variation model applied         Satellite phase centre  calibration    IGS14 phase-centre variation model applied         Satellite trajectories    Best available IGS products.         Earth Orientation              Best available IGS products.          

\vspace{10pt}

         AUSPOS 2.4 Job Number: \# 7513 6 ©Commonwealth of Australia    ser: panzhin@igdure . .  User: panzhin@igduran.ru (Geoscience Australia) 2023 

\vspace{10pt}

\par

\vspace{10pt}

\begin{figure}[h]

\includegraphics[width=\textwidth]{/home/shizlong/spok/pdftolatex/localstore/7513_1699368017513-99999999BBassets/11.jpg}

\centering

\end{figure}

    5.3 Estimation Process         Adjustment    Weighted least-squares algorithm.         Station coordinates    Coordinate constraints are applied at the Reference sites with  standard deviation of Imm and 2mm for horizontal and vertical  components respectively.         Troposphere    Zenith delay parameters and pairs of horizontal delay gradient  parameters are estimated for each station in intervals of 2 hours  and 24 hours.         Ionospheric correction    An ionospheric map derived from the contributing reference sta-  tions is used to aid ambiguity resolution.         Ambiguity              Ambiguities are resolved in a baseline-by-baseline mode using the  Code-Based strategy for 200-6000km baselines, the Phase-Based  L5/L3 strategy for 20-200km baselines, the Quasi-Ionosphere-Free  (QIF) strategy for 20-2000km baselines and the Direct L1/L2  strategy for 0-20km baselines.          

\vspace{10pt}

    5.4 Reference Frame and Coordinate Uncertainty         Terrestrial reference  frame    IGS14 station coordinates and velocities mapped to the mean  epoch of observation.         Australian datums    GDA2020 and GDA94.         Derived AHD    For stations within Australia, AUSGeoid2020 (V20180201) is used  to compute AHD. AUSGeoid2020 is the Australia-wide gravi-  metric quasigeoid model that has been a posteriori fitted to the  AHD. For reference, derived AHD is always determined from the  GDA2020 coordinates. In the GDA94 section of the report, AHD  values are assumed to be identical to those derived from GDA2020.         Above-geoid heights    Earth Gravitational Model EGM2008 released by the National  Geospatial-Intelligence Agency (NGA) EGM Development Team  is used to compute above-geoid heights. This gravitational model  is complete to spherical harmonic degree and order 2159, and con-  tains additional coefficients extending to degree 2190 and order  2159.         Coordinate uncertainty              Coordinate uncertainty is expressed in terms of the 95\% confi-  dence level for GDA94, GDA2020 and ITRF2014. Uncertainties  are scaled using an empirically derived model which is a function  of data span, quality and geographical location.          

\vspace{10pt}

         AUSPOS 2.4 Job Number: \# 7513 7 ©Commonwealth of Australia    ser: panzhin@igdure . .  User: panzhin@igduran.ru (Geoscience Australia) 2023 

\vspace{10pt}

\par

\vspace{10pt}

\end{document}
